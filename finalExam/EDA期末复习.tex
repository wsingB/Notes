% Created 2022-06-23 四 05:16
% Intended LaTeX compiler: xelatex
\documentclass[11pt]{article}
\usepackage{graphicx}
\usepackage{longtable}
\usepackage{wrapfig}
\usepackage{rotating}
\usepackage[normalem]{ulem}
\usepackage{amsmath}
\usepackage{amssymb}
\usepackage{capt-of}
\usepackage{hyperref}
\usepackage{xeCJK}
\author{werbinich}
\date{\today}
\title{}
\hypersetup{
 pdfauthor={werbinich},
 pdftitle={},
 pdfkeywords={},
 pdfsubject={},
 pdfcreator={Emacs 28.1 (Org mode 9.5.2)},
 pdflang={English}}
\begin{document}

\tableofcontents

\section{EDA(电子线路设计自动化)}
\label{sec:orga6d6d82}
\begin{itemize}
\item 补充\footnote{下划线是我补充可能的考点}
\item AD使用辅助设计电路软件
\item 学习AD可以让我们掌握电路设计流程
\end{itemize}
\section{电子设计流程}
\label{sec:org5b9dfaf}
\begin{enumerate}
\item 项目立项
\item 原理图设计
\item PCB建库\footnote{印制电路板}
\item PCB设计
\item 生产文件输出
\item PCB文件加工
\end{enumerate}
\section{Altium Designer 常见文件后缀名}
\label{sec:org44a975c}
\begin{itemize}
\item 工程文件: \textbf{PrjPcb}
\item 元件库文件: \textbf{SchLib}
\item 原理图文件: \textbf{SchDoc}
\item PCB库文件: \textbf{PcbLib}
\item 网络表文件: \textbf{NET}
\item PCB文件: \textbf{PcbDoc}
\end{itemize}
\section{编辑元件属性}
\label{sec:org4a8bc29}
\begin{itemize}
\item \uline{绘制元件库}
\begin{itemize}
\item 多个部分的元件需要add为元件添加新的部分
\item 需要给model添加Footprint等描述
\item 放管脚时注意有x的朝外
\end{itemize}
\item \uline{Designator:元件位号,元件的唯一表示}
\begin{itemize}
\item U?(IC),R?(电阻),C?(电容),J?(接口)\footnote{R:电阻,C:电容,RN:排阻,EC:电解电容,U:芯片,X:晶振,D:二极管,Q:三极管,J:跳线,LED:发光二极管,ZD:整流二极管,FB:磁珠}
\end{itemize}
\item 元件的移动
\begin{itemize}
\item 选择
\begin{itemize}
\item 单选: 鼠标左键
\item 多选: shift+鼠标左键
\end{itemize}
\item 旋转
\begin{itemize}
\item 选中后按空格逆时针旋转
\item Shift+空格顺时针旋转
\item X or Y 根据X或Y进行镜像
\end{itemize}
\end{itemize}
\end{itemize}
\section{原理图绘制}
\label{sec:org4a515e5}
\begin{itemize}
\item \uline{设计方式}
\begin{itemize}
\item 自顶向下
\begin{itemize}
\item 将大模块逐步分解为小模块去设计
\end{itemize}
\item 自底向上
\begin{itemize}
\item 从底层开始设计逐步扩大最后完成
\end{itemize}
\end{itemize}
\item 导线
\begin{itemize}
\item 命令 Place + Wire
\item 功能:
\begin{itemize}
\item 连接电气元件
\item 具有电气特性
\end{itemize}
\item 接地和电源
\begin{itemize}
\item 点击图标或者Place + Power Port
\item 按住Tab可以配置属性
\end{itemize}
\end{itemize}
\item 网络标号
\begin{itemize}
\item 表示多个具有电气意义的导线, 降低原理图复杂度
\item Place + Net Label
\item 同一个网络标号需要完全一致
\item TAA (tools + annotation + annotation schematics)
\begin{itemize}
\item 可以对一个原理图的标号进行编辑和选择
\end{itemize}
\end{itemize}
\item \uline{页连接符}
\begin{itemize}
\item 由于网络符号无法在多张图纸中连接,所以需要使用Port\footnote{端口}进行连接
\item Place Port
\item 其他作了解\footnote{Sheet Entry, Off Sheet Connector, Power Port}
\end{itemize}
\item \uline{总线}
\begin{itemize}
\item Place + Bus
\item 表示具有相同电气意义的一组导线
\item 总线以总线分支引出各条分导线,以网络标号做区分
\begin{itemize}
\item 总线分支 Place + Bus Entry(PU)
\end{itemize}
\end{itemize}
\item No ERC
\begin{itemize}
\item x图标
\item 可以忽略该管脚的错误,双击或者TAB可以修改检查属性
\end{itemize}
\item 辅助线
\begin{itemize}
\item 无电气意义, 用于区分电路的各个部分
\item Place + Line(PDL)
\end{itemize}
\end{itemize}
\section{原理图编译}
\label{sec:org145728b}
\begin{itemize}
\item Project + compile PCB Project xxx.PrjPcb
\item BOM
\begin{itemize}
\item 物料清单表
\item Report + Bill of Materials(RI)
\end{itemize}
\end{itemize}
\section{PCB封装创建}
\label{sec:orgd5bcdd1}
\begin{itemize}
\item 向导法创建
\begin{itemize}
\item 通过对封装类型模板的选择比如DIP对称的封装 \footnote{穿孔,双列}
\item 根据芯片手册填写焊盘参数,一般要比数据手册大一点, 内径,外径
\item 焊盘间距参数:纵向e-2.54mm,横向E1-7.62mm
\item finish
\end{itemize}
\end{itemize}
\section{手工绘制PCB封装}
\label{sec:orgbdc762d}
\begin{itemize}
\item 焊盘
\begin{itemize}
\item Place + Pad
\item 设置形状 \footnote{表贴焊盘需要放在Top layer, 通孔放在Multi-Layer}
\end{itemize}
\item 过孔
\begin{itemize}
\item Place + via
\end{itemize}
\item 放置敷铜
\begin{itemize}
\item Place polygon place
\end{itemize}
\item 网表
\begin{itemize}
\item 网络连接和联系的表示
\item 通过网表连接关系进行PCB的导入
\item Design + Netlist for Project + Protel生成
\end{itemize}
\item \uline{固定孔}
\begin{itemize}
\item 3mm
\item (5mm,5mm)
\end{itemize}
\end{itemize}
\section{PCB布局}
\label{sec:orgfde2e93}
\begin{itemize}
\item \uline{设置PCB板子大小}
\begin{itemize}
\item Q 切换grid(方格)单位 or View + grids + set global snap grids
\item Place line (Keep out layer 或者 Mechanical layer)需要闭合
\item Designer + Board shape + define from objects
\end{itemize}
\item 按照信号走向布局,以每个功能为核心布局
\item PCB类
\begin{itemize}
\item 同一属性的网络或元件或差分放在一起构成一个类别,比如电源,GND,VCC
\item 便与管理和编辑
\item Design + Class 在大类的子类别中使用鼠标右键 add class
\end{itemize}
\item \uline{PCB 规则设置}
\begin{itemize}
\item Clearance 安全距离设计 可以选择规则适配范围\footnote{不同网络,相同网络,所有网络,不同差分}
\item Track是走线 Hole是钻孔 TH Pad通孔焊盘Copper 铜皮
\item enable 启用规则
\item 不要勾选允许短路和开路
\item 设置线宽规则
\end{itemize}
\end{itemize}
\section{一些术语的作用}
\label{sec:org046157c}
\begin{itemize}
\item 泪滴
\begin{itemize}
\item 避免电路板收到巨大外力冲撞时导线与焊盘接触点断开,是的更加美观
\item 保护焊盘避免多次焊接时脱落
\item 信号传输时平滑阻抗,降低急剧跳变
\item tools + teardrops
\end{itemize}
\item 敷铜
\begin{itemize}
\item 增加载流面接和能力
\item 减小底线阻抗,抗干扰
\item 降低压降,提高电源效率
\item 与地线连接,减少环路面积
\item 对称敷铜可以对多层板起到平衡作用
\end{itemize}
\item DRC 检查设计是否满足规则
\begin{itemize}
\item 电源线与接地线要宽一些
\end{itemize}
\item 设置相对原点
\begin{itemize}
\item Edit + Origin + set
\end{itemize}
\item 尺寸标注
\begin{itemize}
\item Place + Dimension linear
\item 便于设计者和生产者获取PCB尺寸以及相关信息
\end{itemize}
\end{itemize}
\section{生产文件输出步骤 Gerber}
\label{sec:orga2d081a}
\begin{itemize}
\item file + fabrication Outputs Gerber
\item \uline{是一个所有电路设计软件都可以生产的模板文件,又叫做光绘文件}
\item 单位:inches
\item 比例:2:4
\item 选择使用的层 Plot Layer used on
\item 丝印层 (GTO\footnote{Gerber Top Overlayer}) 做标识
\item GM1(机械标注层1) GKO(禁止布线层)
\end{itemize}
\end{document}